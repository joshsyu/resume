%%% LaTeX Template: Designer's CV
%%%
%%% Source: http://www.howtotex.com/
%%% Feel free to distribute this template, but please keep the referal to HowToTeX.com.
%%% Date: March 2012


%%%%%%%%%%%%%%%%%%%%%%%%%%%%%%%%%%%%%
% Document properties and packages
%%%%%%%%%%%%%%%%%%%%%%%%%%%%%%%%%%%%%
\documentclass[a4paper,12pt,final]{memoir}

% misc
\renewcommand{\familydefault}{bch}	% font
\pagestyle{empty}					% no pagenumbering
\setlength{\parindent}{0pt}			% no paragraph indentation


% required packages (add your own)
\usepackage{flowfram}										% column layout
\usepackage[top=1cm,left=1cm,right=1cm,bottom=1cm]{geometry}% margins
\usepackage{graphicx}										% figures
\usepackage{url}											% URLs
\usepackage[usenames,dvipsnames]{xcolor}					% color
\usepackage{multicol}										% columns env.
	\setlength{\multicolsep}{0pt}
\usepackage{paralist}										% compact lists
\usepackage{tikz}
\usepackage[style=authoryear,backend=biber]{biblatex}
% Below is for Chinese display
\usepackage{fontspec} % 加這個就可以設定字體
\usepackage{xeCJK} % 讓中英文字體分開設置
\setCJKmainfont{標楷體} % 設定中文為系統上的字型,而英文不去更動,使用原TeX字型
\XeTeXlinebreaklocale "zh"
\XeTeXlinebreakskip = 0pt plus 1pt % 這兩行一定要加,中文才能自動換行

%%%%%%%%%%%%%%%%%%%%%%%%%%%%%%%%%%%%%
% Create column layout
%%%%%%%%%%%%%%%%%%%%%%%%%%%%%%%%%%%%%
% define length commands
\setlength{\vcolumnsep}{\baselineskip}
\setlength{\columnsep}{\vcolumnsep}

% frame setup (flowfram package)
% left frame
\newflowframe{0.2\textwidth}{\textheight}{0pt}{0pt}[left]
	\newlength{\LeftMainSep}
	\setlength{\LeftMainSep}{0.2\textwidth}
	\addtolength{\LeftMainSep}{1\columnsep}
 
% small static frame for the vertical line
\newstaticframe{1.5pt}{\textheight}{\LeftMainSep}{0pt}
 
% content of the static frame
\begin{staticcontents}{1}
\hfill
\tikz{%
	\draw[loosely dotted,color=RoyalBlue,line width=1.5pt,yshift=0]
	(0,0) -- (0,\textheight);}%
\hfill\mbox{}
\end{staticcontents}
 
% right frame
\addtolength{\LeftMainSep}{1.5pt}
\addtolength{\LeftMainSep}{1\columnsep}
\newflowframe{0.7\textwidth}{\textheight}{\LeftMainSep}{0pt}[main01]


%%%%%%%%%%%%%%%%%%%%%%%%%%%%%%%%%%%%%
% define macros (for convience)
%%%%%%%%%%%%%%%%%%%%%%%%%%%%%%%%%%%%%
\newcommand{\Sep}{\vspace{1.5em}}
\newcommand{\SmallSep}{\vspace{0.5em}}

\newenvironment{AboutMe}
	{\ignorespaces\textbf{\color{RoyalBlue} About me}}
	{\Sep\ignorespacesafterend}
\newcommand{\CVSection}[1]
	{\Large\textbf{#1}\par
	\SmallSep\normalsize\normalfont}

\newcommand{\CVItem}[1]
	{\textbf{\color{RoyalBlue} #1}}


%%%%%%%%%%%%%%%%%%%%%%%%%%%%%%%%%%%%%
% Begin document
%%%%%%%%%%%%%%%%%%%%%%%%%%%%%%%%%%%%%
\begin{document}

% Left frame
%%%%%%%%%%%%%%%%%%%%
\begin{figure}
	\hfill
	\includegraphics[width=0.6\columnwidth]{picture}
	\vspace{-7cm}
\end{figure}

\begin{flushright}\small
	徐浩哲(Josh Hsu) \\
	\footnotesize{\url{joshsyu.tw@gmail.com}}  \\
	(0955) 861-987
\end{flushright}\normalsize
\framebreak


% Right frame
%%%%%%%%%%%%%%%%%%%%
\Huge\bfseries {\color{RoyalBlue} 徐浩哲} \\ % Your name
\Large\bfseries  Software Engineer \\

\normalsize\normalfont

% About me
%\begin{AboutMe} \\
大學畢業於電機工程學系,接觸過電路設計(VLSI ),系統設計(人臉辨識系統),
和嵌入式開發,因特別喜歡研究系統程式,
研究所轉念資訊工程,開發系統模擬器,本身的興趣有:\\
系統與架構開發,嵌入式系統,虛擬化技術,高性能計算,
IC設計,智慧駕駛

%\end{AboutMe}

% Education
\CVSection{Education}
\CVItem{2015 Feb. - 2016 Jun.(Expected), 國立清華大學}\\
GPA 4.1/4.3 \\
資訊工程碩士 (論文指導教授:鍾葉青 教授) \\
系統軟體實驗室
\SmallSep

\CVItem{2011 Sep.- 2015 Feb, 國立中正大學}\\
GPA 4.51/4.3 \\
7 次書卷獎/ 7 學期\\
電機工程學士(專題指導教授:葉經緯 教授)\\
晶片系統組 
\SmallSep

\CVItem{2008 Sep. - 2011 Jun., 國立嘉義高中}\\
\Sep

% Experience
\CVSection{Experience}
\CVItem{Feb. 2015 - present, HSAemu}\\
系統架構模擬器\\
HSA Foundation 成員之一\\
聯發科產學合作計畫與國科會深耕計畫之一\\
基於QEMU 開源專案,實作符合HSA 規格書之硬體架構模擬器\\
並支援OpenCL 2.0 運算平台\\
\SmallSep
\CVItem{Feb. 2015 - present, PQEMU}\\
平行化模擬器\\
基於QEMU 開源專案,實作平行架構的軟體模擬器\\
\SmallSep
\CVItem{Feb. 2015 - present, Family Monitoring System}\\
監視影像串流\\
透過嵌入式開發版與視訊鏡頭串流影像鏡頭至網頁\\
\SmallSep
\CVItem{Feb. 2015 - present, Github Pages}\\
靜態個人網頁\\
一開始使用Jekyll 做為平台現在轉移至Hugo\\
\SmallSep
\CVItem{Feb. 2014 - Feb. 2015, Face-tracking System}\\
人臉辨識系統\\
使用OpenCV 模組訓練機器便是人臉\\
\SmallSep
\CVItem{Feb. 2013 - Feb. 2014, Design Controller Chip}\\
VLSI 電路設計\\
從Verilog 設計控制晶片到最後合成電路\\
\SmallSep
\CVItem{Sep. 2013 - Jun. 2014, Tiny Basic on 8051}\\
單晶片開發\\
在單晶片8051 上透過組合語言實作Tiny Basic 語言\\
\Sep

\newpage
% Left frame
%%%%%%%%%%%%%%%%%%%%
%\begin{figure}
%	\hfill
%	\includegraphics[width=0.6\columnwidth]{picture}
%	\vspace{-7cm}
%\end{figure}

\begin{flushright}\small
	徐浩哲(Josh Hsu) \\
	\footnotesize{\url{joshsyu.tw@gmail.com}}  \\
	(0955) 861-987
\end{flushright}\normalsize
\framebreak
% Skills
\CVSection{Skills}
\CVItem{Platforms}
\begin{multicols}{3}
\begin{compactitem}[\color{RoyalBlue}$\circ$]
	\item Linux
	\item FreeBSD 
	\item Docker
	\item Qemu
	\item VMware
	\item VirtualBox
\end{compactitem}
\end{multicols}
\SmallSep

\CVItem{Computer Languages}
\begin{multicols}{3}
\begin{compactitem}[\color{RoyalBlue}$\circ$]
	\item C/C++ 
	\item Python 
	\item Go Language
	\item Assembly
	\item Verilog
\end{compactitem}
\end{multicols}
\SmallSep
\CVItem{Familiar Development Tools}
\begin{multicols}{3}
\begin{compactitem}[\color{RoyalBlue}$\circ$]
	\item GNU Tools
	\item EDA Tools
	\item GDB
	\item Git
	\item Latex
	\item Markdown
%	\item \ldots
\end{compactitem}
\end{multicols}
\CVItem{Familiar Development Packages}
\begin{multicols}{3}
\begin{compactitem}[\color{RoyalBlue}$\circ$]
	\item OpenCL
	\item CUDA
	\item OpenCV
	\item Pthread
	\item MPI
	\item OpenMP
	\item Jekyll
	\item Hugo
\end{compactitem}
\end{multicols}
\CVItem{Familiar Areas}
\begin{multicols}{3}
\begin{compactitem}[\color{RoyalBlue}$\circ$]
	\item Virtualization
	\item Embedded System
	\item Open Source
	\item High performance computing
\end{compactitem}
\end{multicols}
\Sep 
% Publication
%\CVSection{Publishcation}


%\CVSection{Something other}
%Lorem ipsum dolor sit amet, consectetur adipiscing elit. Vivamus vel bibendum metus. Proin rutrum pharetra molestie. Cras sollicitudin nulla nec leo lobortis in tristique purus pretium. Ut eu felis nulla. Pellentesque condimentum justo ut ligula feugiat nec facilisis tellus ultricies. Nullam sit amet dictum ipsum. Sed lacus neque, hendrerit eu rhoncus nec, pellentesque vitae sem.
%\Sep
%\clearpage
%\framebreak
%\framebreak

%\CVSection{Something else}
%Lorem ipsum dolor sit amet, consectetur adipiscing elit. Vivamus vel bibendum metus. Proin rutrum pharetra molestie. Cras sollicitudin nulla nec leo lobortis in tristique purus pretium. Ut eu felis nulla. Pellentesque condimentum justo ut ligula feugiat nec facilisis tellus ultricies. Nullam sit amet dictum ipsum. Sed lacus neque, hendrerit eu rhoncus nec, pellentesque vitae sem.
%\Sep

% References
%\CVSection{References}
%References upon request.

%%%%%%%%%%%%%%%%%%%%%%%%%%%%%%%%%%%%%
% End document
%%%%%%%%%%%%%%%%%%%%%%%%%%%%%%%%%%%%%
\end{document}
